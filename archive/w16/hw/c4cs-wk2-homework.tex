\documentclass{article}
%\usepackage{fullpage}
%\usepackage[top=1in, bottom=1in, left=1cm,right=1cm]{geometry}
\usepackage[top=.75in, bottom=.75in, left=.75in,right=.75in]{geometry}
\usepackage{fancyhdr}
\usepackage{lastpage}
\usepackage{color}
\usepackage[colorlinks,urlcolor={blue}]{hyperref}

\begin{document}

\fancyfoot[L]{C4CS -- W'16}
\fancyfoot[R]{Revision 1.0}
\fancyfoot[C]{\thepage~/~\pageref*{LastPage}}
\pagestyle{fancyplain}


\title{\textbf{Homework 2\\Shells and Your Environment}}
\author{Assigned: Friday, January 15, 11:00AM}
\date{\textbf{\color{red}{Due: Friday, January 22, 11:00AM (Hard Deadline)}}}
\maketitle


\section*{Submission Instructions}
Submit this assignment on \href{https://gradescope.com/courses/2248}{Gradescope}.
You must submit every page of this PDF.
We recommend using the free online tool \href{https://www.pdfescape.com}{PDFescape}
to edit and fill out this PDF.
You may also print, handwrite, and scan this assignment.


\section{Understanding your environment}

When you open a new terminal window, a lot has actually happened behind the
scenes before your prompt shows up. In lecture we introduced the files
\texttt{\textasciitilde/.bashrc} and \texttt{\textasciitilde/.bash\_aliases}
as some files that are read when a new session starts, however, there more
than just those.

\textbf{List all of the files read by your shell during startup that
contribute to your environment. List files in the order that they are read.}
\emph{Hint: Don't forget about global files, such as those in \texttt{/etc}}

\vspace{4cm}


\noindent
Run the command \texttt{set}.\footnote{
  That dumps a lot of text to the screen. Try running \texttt{set | less} to
  get a scrollable view or \texttt{set $>$ output.txt} to save the output to a
  file for easier viewing.
}
This command prints your current
environment, that is, all of the variables currently set to any
value.  Also try the command \texttt{env}, which only includes variables
that have been ``exported'', that is variables that will be set for child
processes. Why do you think so many variables are set, but not exported?

\textbf{Pick a variable that is set but not exported (something printed by
  \texttt{set} but not by \texttt{env}). Explain why you think that
  variable is useful for the shell process, but not for a child process
  spawned (created) by the shell:
}
\vspace{2cm}


\noindent
While some variables should be familiar from the first part of this question,
there are many more variables in your environment than those set by
\texttt{.bashrc} and friends.

\textbf{Pick a variable not set during shell startup. Explain where that
variable comes from (what sets it) and what its value means
\emph{(do not use the same variable as the previous question)}:}


\newpage
\section{Special Variables}

Bash has quite a few special variables that can be very useful when writing
scripts or while working at the terminal.

\textbf{What does the variable \texttt{\$?} do? Give an example where this
value is useful}
\vspace{3cm}

\textbf{What does the variable \texttt{\$\_} do? Give an example where this
value is useful}
\vspace{3cm}


\section{Understanding your \texttt{PATH}}

In a terminal, type \texttt{PATH=} (just hit enter after the equal sign, no
space characters anywhere). Try to use the terminal like normal (try running
\texttt{ls}). What happened?

\textbf{Give an example of a command that used to work but now doesn't:}
\vspace{3cm}

\textbf{Can you still run this command with an empty \texttt{PATH}? How?}
\vspace{3cm}

\textbf{Give an example of a command that works the same even with an empty
\texttt{PATH}. Why does this command still work?}
\vspace{3cm}


\newpage
\section{Basic Scripting}

Recall from lecture that scripting is really just programming, only in a very
high-level language. Interestingly, \texttt{sh} is probably one of the oldest
languages in regular use today.

\texttt{make} is a good tool for build systems, but we can actually use some
basic scripting to accomplish a lot of the same things.
First, write a simple C program that prints ``Hello World!''.
Write a shell script named \texttt{build.sh} that performs the following
actions:
\begin{enumerate}
  \item Compile your program
  \item Runs your program
  \item Verifies that your program outputs exactly the string ``Hello World!''
  \item Prints the string ``All tests passed.'' if the output is correct, or
    prints ``Test failed. Expected output $>>$Hello World$<<$, got
    output $>>$\{the program output\}$<<$''.
\end{enumerate}

\textbf{Copy the output of \texttt{cat build.sh} here:}


\end{document}
