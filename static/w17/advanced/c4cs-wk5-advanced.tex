\documentclass{article}
\usepackage{amssymb}
\usepackage{comment}
\usepackage{courier}
\usepackage{fancyhdr}
\usepackage{fancyvrb}
\usepackage[T1]{fontenc}
\usepackage[top=.75in, bottom=.75in, left=.75in,right=.75in]{geometry}
\usepackage{graphicx}
\usepackage{lastpage}
\usepackage{listings}
\lstset{basicstyle=\small\ttfamily}
\usepackage{mdframed}
\usepackage{parskip}
\usepackage{ragged2e}
\usepackage{soul}
\usepackage{upquote}
\usepackage{xcolor}

% http://www.monperrus.net/martin/copy-pastable-ascii-characters-with-pdftex-pdflatex
\lstset{
upquote=true,
columns=fullflexible,
literate={*}{{\char42}}1
         {-}{{\char45}}1
         {^}{{\char94}}1
}
\lstset{
  moredelim=**[is][\color{blue}\bf\small\ttfamily]{@}{@},
}

% http://tex.stackexchange.com/questions/40863/parskip-inserts-extra-space-after-floats-and-listings
\lstset{aboveskip=6pt plus 2pt minus 2pt, belowskip=-4pt plus 2pt minus 2pt}

\usepackage[colorlinks,urlcolor={blue}]{hyperref}

\begin{document}

\fancyfoot[L]{\color{gray} C4CS -- W'17}
\fancyfoot[R]{\color{gray} Revision 1.0}
\fancyfoot[C]{\color{gray} \thepage~/~\pageref*{LastPage}}
\pagestyle{fancyplain}


\title{\textbf{Advanced Exercise -- Week 5\\}}
\author{\textbf{\color{red}{Due: Before February 18, 10:00PM* (ish, see below)}}}
\date{}
\maketitle


\section*{Contributing to an open source project}

While version control is useful for your own projects and helps with group
class projects, what really makes it awesome it how accessible it makes
collaborating and contributing to big (or small) projects.

One thing that's awesome about computer science is that there are many
opportunities for early-career students to make contributions -- if you aren't
familiar, check out
\href{https://www.quora.com/What-companies-have-internship-programs-specifically-for-college-freshmen-and-sophomores}{some
  of the programs for CS freshman/sophomores at major companies}; or check out
alternative programs like Google Summer of Code
(\href{http://write.flossmanuals.net/gsocstudentguide/am-i-good-enough/}
{``Am I Good Enough?''}, the summer of code program can be an excellent way to
spend a summer in place of a more traditional internship.
\href{https://developers.google.com/open-source/gsoc/timeline}
  {Applications open March~20 and close April~3 this year})

Slightly less committal than a full-time job, you're also more than qualified
to help contribute to open source projects. Sometimes this is contributing
code. Many times the most helpful can be improving documentation -- try to
follow the installation instructions and take notes on what confused you and
how might improve them.

For this exercise, the goal is to make a \ul{non-trivial} contribution to an
open source project.

\begin{quote}
  \emph{non-trivial}, adj. -- A loosely defined concept. More than just fixing
  the spelling of a few words. Rewriting a paragraph of documentation to be
  clearer is great though. A code fix could be a one character change if it
  fixes a bug, etc. Ultimately, it'll be up to the discretion of course staff
  who will \href{https://en.wikipedia.org/wiki/I_know_it_when_I_see_it} {know
    it when they see it}, but the goal is to be reasonable. Something
  non-trivial should take more than two minutes, probably doesn't need to take
  more than fifteen.
\end{quote}

\section*{Submission Instructions}

Since you're sometimes working on someone else's schedule (i.e. don't push
another project to merge your changes just to meet the deadline for your
homework!), you need to have a \ul{substantial portion} of the work done by
the deadline.

\begin{quote}
  \emph{substantial}, adj. -- A loosely defined concept. Basically a
  well-formed commit or series of commits, an explanation to the project
  maintainer about what you'd like to change/fix and why, and submitted to the
  project. This should be something publically viewable / auditable (e.g. a
  pull request, a post to a mailing list, etc).

  Again, we'll know it when we see it :).
\end{quote}

It's possible that the project will reject your change, sometimes not for a
great reason. If you had what we deem to be a well-crafted request, we'll give
credit even it was not merged. The goal is to get you some experience with
learning how to submit changes to another project.

\begin{mdframed}\centering
A good first place to start is by reading over
\href{https://guides.github.com/activities/contributing-to-open-source/#contributing}{GitHub's
  guide to contributing to open source projects}.
\end{mdframed}

\newpage

\section*{One Option: Contributing to the class website}

This class itself is an open source project!

If you haven't already, check out \url{https://c4cs.github.io/reference}

The goal is to build up a quick reference for common commands, as well as some
examples of how to use them and any gotcha's they may have. The hope is that
it's easier for people who are learning (i.e.\ you) to write documentation
that's helpful for other people who are learning as opposed to course staff
who don't really remember well what was confusing when we were first learning.

Look through the reference and find something that could use improvement. Or,
add a new command that we've talked about in class or that you have other
experience with that's not in the reference yet.

To ensure that people aren't working on the same things, before you get
started, check to see if there's an
\href{https://github.com/c4cs/c4cs.github.io/issues}{open issue} for what you
want to work on. If no one's doing it yet, make a new issue so that others
know what you're working on. If someone on the course staff opened an issue for
something you'd like to work on please comment to claim.

\subsection*{Some help to get started}

\subsubsection*{GitHub Pages}

Notice how the course homepage is \texttt{c4cs.\ul{github.io}}? GitHub has a
really neat feature called \href{https://pages.github.com/}{GitHub Pages},
that will turn a repository into a website -- hosted for free.
What's really nice about this is that it means it is very easy for many
people to collaborate to develop a website, it's just a repository!

In the simplest setup, GitHub will simply serve the files in the repository as
static web pages. Writing lots of HTML by hand, however, can be a pain, so
GitHub supports a \emph{site generator}, \href{https://jekyllrb.com/}{jekyll}.
Using jekyll, adding an update to the homepage is as simple as adding
\href{https://github.com/c4cs/c4cs.github.io/blob/master/_updates/f16/2016-10-11-chaos.md}{a text file}.
The site will also
\href{https://github.com/c4cs/c4cs.github.io/blob/master/index.html#L32}%
{automatically hide updates older than one month}.

\subsubsection*{Getting Going}

First, grab a copy of course website repository
(\url{https://github.com/c4cs/c4cs.github.io}). From there, crack open the
\texttt{README} file. This is almost always a good place to start. It usually
contains background on the project, some information regarding installation (if
it's a package), dev environment setup, and anything else that might be
pertinent to someone just diving into a new codebase.

Once things are running, it should print out
\begin{quote}\tt
  Server address: http://127.0.0.1:4000/
\end{quote}

Visit \url{http://127.0.0.1:4000/} in your browser and you should see your own
local copy of the course website! Try making some changes to the site and then
refresh the page in your browser.

\bigskip
\bigskip
\bigskip
\bigskip

{\centering
  \hrule
\bigskip
  \LARGE OR \\
  \Large (next page)
\bigskip
  \hrule
}


\newpage

\section*{Another Option: Any project of your choosing}


\subsection*{How to Get Started}

Getting started is the hardest part. There is an overwhelming number of
projects to choose from.
Big projects, such as Mozilla, often have
\href{https://developer.mozilla.org/en-US/docs/Introduction}
{documentation explaining how to contribute},
their bug trackers will track and let you query things like
\href{https://bugzilla.mozilla.org/buglist.cgi?resolution=---&classification=Client%20Software&emailtype1=regexp&status_whiteboard_type=allwordssubstr&query_format=advanced&emailassigned_to1=1&status_whiteboard=%5Bgood%20first%20bug%5D&email1=nobody}
{[good first bug]},
and their contributors are often friendly to beginners, with
\href{https://bugzilla.mozilla.org/show_bug.cgi?id=361983#c11}
{good advice for getting started}.
The negative with big projects is that they are big. You have to learn a large
codebase in order to get started at all.

On the flip side, small projects, such as
\href{https://github.com/kylelady/omnomnorth}{omnomnorth}
can be easier to make changes to, but harder to get started with as they do
not have mature ``getting started'' guides.

If possible, I recommend picking something that you use. It's easier to
motivate yourself to work on something you use, and there's something pretty
cool about \href{https://github.com/um-cseg/chez-betty/}{using your own
software every day}.
\href{https://github.com/Aluxian/Facebook-Messenger-Desktop}{Many}
\href{https://github.com/GNOME/gnome-terminal}{things}
\href{https://github.com/gnachman/iTerm2}{you}
\href{https://github.com/tmux/tmux}{(should)}
\href{https://trac.transmissionbt.com/wiki/Building}{use}
\href{https://github.com/scummvm/scummvm}{are}
\href{https://github.com/GNOME/gimp}{open}
\href{https://github.com/videolan/vlc}{source}.
Many libraries, such as the
\href{https://bitbucket.org/birkenfeld/pygments-main}
{near-universal syntax highlighting library}, the
\href{https://github.com/jgm/pandoc}
{near-universal document converter}, or a
\href{https://github.com/tartley/colorama}
{simple library for pretty terminal output}
have a lot of low-hanging fruit (check out the ``Issues'' tab).

\subsection*{Some local options}

Are you a fan of \href{https://chezbetty.eecs.umich.edu}{Chez Betty}? There
are some open \href{https://github.com/um-cseg/chez-betty/issues}{issues} there.
Speak a foreign language? Help extend Betty's
\href{https://github.com/um-cseg/chez-betty/blob/master/README.translation.md}{translation
support} (or help translate any other project, there's huge demand for
translations).

\href{https://eecs.help}{eecs.help}, our office hours help queue is also
\href{https://github.com/mterwill/classroom-help-queue/issues}{open source}.

Interested in getting into research? Some of the research labs at Michigan
have big projects that could use help. Check out Alex Halderman research group
projects: \href{https://github.com/zmap}{ZMap} and
\href{https://github.com/Censys}{Censys}. Or stuff from my lab,
\href{https://github.com/lab11}{Lab11}, the embedded systems research group.
Or Ed Olsen's \href{https://april.eecs.umich.edu/software/}{Robotics group}.
If you're interested in research with someone, fixing one of their problems is
a \emph{great} way to introduce yourself :).

\end{document}
